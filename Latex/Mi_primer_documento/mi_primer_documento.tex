\documentclass{article} %es necesario indicar el tipo de documento que queremos crear.
% Otras opciones son: book y report
%%%%%%%%%%%%%%%%%%%%%%%%%%%%%
% 			PREAMBULO
%%%%%%%%%%%%%%%%%%%%%%%%%%%%%

%\usepackage[scale=.85]{geometry}


%\usepackage{amssymb, amsmath} %Paquetes matemáticos de la American Mathematical Society

%\usepackage[spanish]{babel} % Indica el idioma de escritura del dumento.
%\usepackage[utf8]{inputenc} % paquete de caracteres que permita escribir caracteres especiales en LaTeX

\title{Mi t\'itulo}
\date{\today}
\author{Pablo Jorge Her\'andez Hernandez}

\usepackage{graphicx}

\begin{document}
\tableofcontents
%\maketitle %Para agregar una pequeña portada



%%%%%%%%%%%%%%%%%%%%%%%%%%%%%
%  CREANDO UNA PORTADA USANDO EKL ENTORNO TITLEPAGE
%%%%%%%%%%%%%%%%%%%%%%%%%%%%%
%
%\begin{titlepage}
%\centering
%{\includegraphics[width=0.35\textwidth]{UMAR.png}\par}
%{\bfseries\LARGE Universidad del Mar\par}
%\vspace{1cm}
%{\scshape\Large Licenciatura en Actuaría\par}
%\vspace{2cm}
%{\scshape\Huge T\'itulo \par }
%\vspace{2cm}
%{\itshape\Large Proyecto....\par}
%\vfill
%{\Large Autor: \par}
%{\Large Nombre Apellidos \par}
%\vfill
%{\Large \today \par}

%Cada línea de texto termina con el comando \par. Esto indica que debe crearse un nuevo párrafo. De lo contrario, LaTeX generaría todos los elementos de texto en una sola línea.
%\end{titlepage}%

\section{Mi primer seccion}
\subsection{Mi primera subseccion}
\subsubsection{Mi primera subseccion}
%\section{Mi segunda seccion}

Este es mi primer documento en \LaTeX!\\

Esta es mi segunda l\'inea de 


\end{document}